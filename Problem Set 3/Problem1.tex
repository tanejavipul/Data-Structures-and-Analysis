\documentclass{article}
\usepackage[utf8]{inputenc}

\title{CSC263: Problem Set 3}
\date{October 8, 2019}

\usepackage[utf8]{inputenc}
\usepackage{graphicx}
\usepackage{listings}
\usepackage{xcolor}
\usepackage{natbib}
\usepackage{graphicx}
\usepackage{amsmath}
\usepackage{enumitem}

\definecolor{codegreen}{rgb}{0,0.6,0}
\definecolor{codegray}{rgb}{0.5,0.5,0.5}
\definecolor{codepurple}{rgb}{0.58,0,0.82}
\definecolor{backcolour}{rgb}{0.95,0.95,0.92}

\lstdefinestyle{mystyle}{
    backgroundcolor=\color{backcolour},   
    commentstyle=\color{codegreen},
    keywordstyle=\color{magenta},
    numberstyle=\tiny\color{codegray},
    stringstyle=\color{codepurple},
    basicstyle=\ttfamily\footnotesize,
    breakatwhitespace=false,         
    breaklines=true,                 
    captionpos=b,                    
    keepspaces=true,                 
    numbers=left,                    
    numbersep=5pt,                  
    showspaces=false,                
    showstringspaces=false,
    showtabs=false,                  
    tabsize=2
}
 
\lstset{style=mystyle}

\begin{document}

\maketitle

\section{Problem 1}

\begin{enumerate}[label=(\alph*)]

\item \begin{figure}[htp]
    \centering
    \includegraphics[width=13cm]{HBalancedTree.png}
    \caption{Height balanced tree that is not ideally height balanced.}
    \label{fig:tree}
\end{figure}

This is an example of a height balanced tree that is not ideally height balanced. By definition an ideally height balanced tree has every leaf at depth $h$ or $h-1$, and every node of depth less than $h-1$, has 2 children, and clearly Node C has only 1 child, Node G. Note that in this case, the tree in figure 1 has a height of 4 (assuming the root has a height of 1), and Node F has a depth of 4. Node C is at a depth of 2 which is less than $h - 1 = 4 - 1 = 3$. 

However, this tree is height balanced. This is because for every node the height of the left subtree is within $\pm 1$ of the height of the right subtree. To see that this is true for each node, we use balance factors. Balance Factor = $h(R) - h(L)$, where $h(R)$ and $h(L)$ stands for height of right subtree and height of left subtree respectively. Thus, if the balance factor is one of $-1$, $0$, $1$, then the tree is height balanced. As all balance factors in figure 1 are either $-1$, $0$, or $1$, the tree is height balanced.

\end{enumerate}

\end{document}
