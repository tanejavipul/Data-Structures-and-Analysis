\documentclass{article}
\usepackage[utf8]{inputenc}

\title{CSC263: Problem Set 2}
\date{September 24, 2019}


\usepackage{listings}
\usepackage{xcolor}
\usepackage{natbib}
\usepackage{graphicx}
\usepackage{amsmath}
\usepackage{enumitem}

\definecolor{codegreen}{rgb}{0,0.6,0}
\definecolor{codegray}{rgb}{0.5,0.5,0.5}
\definecolor{codepurple}{rgb}{0.58,0,0.82}
\definecolor{backcolour}{rgb}{0.95,0.95,0.92}

\lstdefinestyle{mystyle}{
    backgroundcolor=\color{backcolour},   
    commentstyle=\color{codegreen},
    keywordstyle=\color{magenta},
    numberstyle=\tiny\color{codegray},
    stringstyle=\color{codepurple},
    basicstyle=\ttfamily\footnotesize,
    breakatwhitespace=false,         
    breaklines=true,                 
    captionpos=b,                    
    keepspaces=true,                 
    numbers=left,                    
    numbersep=5pt,                  
    showspaces=false,                
    showstringspaces=false,
    showtabs=false,                  
    tabsize=2
}
 
\lstset{style=mystyle}

\begin{document}

\maketitle

\section{Problem 1}

\begin{enumerate}[label=(\alph*)]

\item Let $Q$ be an array whose indices start from 1. Assume all keys in $Q$ are distinct integers.


\begin{lstlisting}[language=Python]
ExtractSecondLargest(Q):
    if Q.length < 2 then
        return None
    else then
        first = EXTRACT_MAX(Q)
        second = EXTRACT_MAX(Q)
        INSERT(Q, first)
        return second
\end{lstlisting}

\item Assume the length of array, $Q$, is greater than 2. Then we enter the else conditional on line 5 of the code. Line 5 of the code first extracts the element of $Q$ with the largest key, and stores it in the variable $first$. Line 6 of the code, extracts the element with the largest key of the modified array, $Q$, and stores it in the variable $second$. We note here that $second$ is the element of $Q$ with the second largest key, when compared with the original $Q$. Note that at this point, $Q$ is missing two elements with the largest keys, since they were extracted in lines 5 and 6. Then on line 7, the INSERT function, inserts the element with the largest key from the original $Q$ back into $Q$. Thus, the resulting $Q$ is only missing the element with the second largest key. In Line 8 of the code, the element with the second largest key is returned, thus it has been extracted from $Q$.

We note that the heap structure is preserved at every line because of the functions used to create $EXTRACTSECONDLARGEST(Q)$. First, we focus on the $EXTRACT\_MAX(Q)$ function. This function preserves the heap structure through the $MAX\_HEAPIFY(Q,i)$ function, where $i$ is an index into the array. Once the maximal element has been extracted, the $MAX\_HEAPIFY(Q,i)$ function works by determining the largest of elements $Q[i]$, $Q[LEFT(i)]$, and $Q[RIGHT(i)]$, and the largest element's index is stored. If $Q[i]$ is the largest, then the subtree rooted at node $i$ is already a max heap and so the function terminates. Otherwise, if one of the two children are larger then $Q[i]$ is swapped with the largest of its' children ($Q[LEFT(i)]$ or $Q[RIGHT(i)]$). This causes the original node at index $i$ and its children to satisfy the max-heap property. However, the node indexed by one of the children now has the original value $Q[i]$ and the subtree now rooted at one of the children could possibly violate the max-heap property and so $MAX\_HEAPIFY$ is called recursively on that subtree. In this way, the heap structure is maintained.    

Next, we focus on the $INSERT(Q,x)$ function, where $x$ is an element that is to be inserted into the array, $Q$. The element is inserted at the very end of the array, i.e. a new leaf to the tree, but because of the $INCREASE\_KEY(Q,i,key)$ (where $i$ is an index into the array) function call inside $INSERT(Q,x)$, the heap structure is maintained because $INCREASE\_KEY(Q,i,key)$ sets the key of the new node to its correct value and thus maintains the max-heap property.

In this way, at the end of each line of code that is executed in the else condition, the heap structure is maintained.

\item Suppose $Q$ has a length of $n > 2$. Then we skip lines 2 and 3 of the code which run in constant time and enter the else conditional. From lecture and the textbook, we know that $EXTRACT\_MAX(Q)$ and $INSERT(Q,x)$ run in $O(log(n))$ time (see pages 163 and 164 from CLRS). Thus, from the pseudo-code we can see that on lines 5 and 6 we run $EXTRACT\_MAX(Q)$ twice and on line 7 $INSERT(Q,x)$, thus we get a run-time of $clog(n)$, where $c$ is some constant. Accounting for other lines of code like the return statement on line 8, we get $T(n) = clog(n) + d$ $\epsilon$ $O(log(n))$, for some constant $d$.  

\end{enumerate}

\end{document}
